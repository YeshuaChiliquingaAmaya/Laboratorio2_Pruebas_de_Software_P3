\documentclass[12pt, a4paper]{article}
\usepackage[spanish]{babel}
\usepackage[utf8]{inputenc}
\usepackage[T1]{fontenc}
\usepackage{graphicx}
\usepackage{float}
\usepackage{enumitem}
\usepackage{booktabs}
\usepackage{array}
\usepackage{geometry}
\usepackage{helvet}
\usepackage{colortbl}
\usepackage{amssymb}  % Para símbolos adicionales como \checkmark
\usepackage{url}  % Para formatear URLs en la bibliografía
\usepackage[12pt]{extsizes}  % Para tamaños de letra personalizados
\usepackage{anyfontsize}     % Para cualquier tamaño de letra
\usepackage{fancyhdr}        % Para controlar encabezados y pies de página
\pagestyle{empty}           % Elimina la numeración de páginas

% Configuración de tamaños de sección
\usepackage{titlesec}
\titleformat*{\section}{\fontsize{12}{14}\selectfont\bfseries}
\titleformat*{\subsection}{\fontsize{11}{13}\selectfont\bfseries}
\titleformat*{\subsubsection}{\fontsize{11}{13}\selectfont\bfseries\itshape}

% Configuración de la página
\geometry{
    a4paper,
    left=1.5cm,    % Margen izquierdo reducido
    right=1.5cm,   % Margen derecho reducido
    top=2cm,       % Margen superior
    bottom=2cm,    % Margen inferior
}

% Usar Helvetica (similar a Arial) como fuente principal
\renewcommand{\familydefault}{\sfdefault}

% Definir colores
\definecolor{graycell}{RGB}{217, 217, 217} % #d9d9d9

\begin{document}

% Tabla con estructura exacta
\begin{table}[h]
    \centering
    \arrayrulecolor{black}
    \setlength{\arrayrulewidth}{0.5pt}
    \renewcommand{\arraystretch}{1.5}
    \small
    
    % Definición de columnas con anchos ajustados
    \begin{tabular}{|p{3.5cm}|p{4.8cm}|p{2.0cm}|p{1.8cm}|p{2.0cm}|p{1.8cm}|}
    \hline
    % Fila 1
    \rowcolor{graycell}\textbf{DEPARTAMENTO:} & 
    \cellcolor{white}\textbf{CIENCIAS DE LA COMPUTACIÓN} & 
    \textbf{CARRERA:} & 
    \multicolumn{3}{p{5.6cm}|}{\cellcolor{white}\textbf{INGENIERÍA DE SOFTWARE}} \\
    \hline
    % Fila 2
    \rowcolor{graycell}\textbf{ASIGNATURA:} & 
    \cellcolor{white}Pruebas de Software & 
    \textbf{NIVEL:} & 
    \cellcolor{white}6to & 
    \cellcolor{graycell}\textbf{FECHA:} & 
    \cellcolor{white}01/08/25 \\
    \hline
    % Fila 3
    \rowcolor{graycell}\textbf{DOCENTE:} & 
    \cellcolor{white}Ing. Luis Castillo, Mgtr. & 
    \textbf{PRÁCTICA N°:} & 
    \cellcolor{white}1 & 
    \cellcolor{graycell}\textbf{CALIF.:} & 
    \cellcolor{white} \\
    \hline
    \end{tabular}
\end{table}
\vspace{1cm}

% Título (TL 14):Título o tema de la práctica con una extensión máxima de 20 palabras. Tamaño de letra (TL) 14

\begin{center}
    \fontsize{14}{16}\selectfont\textbf{Aplicación de pruebas del sistema reservas}
\end{center}

% Nombre del estudiante (TL 12)
\begin{center}
    \fontsize{12}{14}\selectfont\textbf{Yeshua Amador Chiliquinga Amaya}
    \vspace{0.5cm}
    \rule{\textwidth}{0.5pt}
\end{center}

% Resumen (TL 11): Se expone de manera clara lo realizado en el laboratorio, su propósito y las conclusiones a las que se llegó, se recomienda mínimo 8 líneas y máximo 15 líneas. (TL11)

\begin{center}
    \fontsize{11}{13}\selectfont\textbf{RESUMEN}
    
    \vspace{0.5cm}
    \parbox{0.9\textwidth}{\centering
    
    }
\end{center}
\vspace{0.5cm}

%Palabras Claves: se escriben las palabras más importantes de la práctica o experimento (máximo tres).

\textbf{Palabras Claves:} 

\vspace{1cm}
    
%Se incluyen aspectos relacionados con los objetivos, resaltando la realización de las actividades en función al manejo y disciplina en el laboratorio.
\section*{1. INTRODUCCIÓN}


%2.1	Describir los alcances o metas de la práctica
\section*{2. OBJETIVOS}

\subsection*{2.1 Objetivo General}


\subsection*{2.2 Objetivos Específicos}
\begin{itemize}
    \item 
\end{itemize}

%Dependiendo de las necesidades se describen los conceptos de las herramientas, tecnologías, frameworks, librerías, etc., utilizados en la práctica de laboratorio.

\section*{3. MARCO TEÓRICO}

\subsection*{3.1 Entorno de Desarrollo}
\begin{itemize}
    \item 
\end{itemize}

%Se explicarán el desarrollo de la práctica o experimento realizado en clase con el docente.
\section*{4. DESCRIPCIÓN DEL PROCEDIMIENTO}

%Se debe explicar las preguntas y/o actividades prácticas extra enviadas por el docente.
\section*{5. PREGUNTAS/ACTIVIDADES}

%Sintetizar los resultados de acuerdo a los objetivos planteados.
\section*{6. CONCLUSIONES}

\begin{itemize}[leftmargin=*]
    \item 
\end{itemize}

%Recomendaciones de acuerdo a los objetivos planteados
\section*{7. RECOMENDACIONES}

\begin{itemize}[leftmargin=*]
    \item 
\end{itemize}

%Autor o autores. Año. Título del artículo, Nombre de la Revista, Editorial, Páginas o ubicación de la consulta. Fecha de consulta
\section*{8. BIBLIOGRAFÍA}

\noindent\hangindent=1.5em
[1] OWASP Foundation, ``OWASP Testing Guide v4.2'', 2021. [En línea]. Disponible: \url{https://owasp.org/www-project-web-security-testing-guide/}

\end{document}